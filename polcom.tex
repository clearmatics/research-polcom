% !TeX spellcheck = en_UK
% \let\accentvec\vec              
\documentclass[runningheads,11pt]{llncs}
%\let\spvec\vec \let\vec\accentvec
\newcommand\hmmax{0}
\newcommand\bmmax{0}

\usepackage{amssymb,amsmath}
%\let\vec\spvec
%\usepackage{lmodern}
\usepackage{newtxmath,newtxtext}
\usepackage[T1]{fontenc}
% \def\vec#1{\mathchoice{\mbox{\mathbf$\displaystyle#1$}}
% {\mbox{\mathbf$\textstyle#1$}} {\mbox{\mathbf$\scriptstyle#1$}}
% {\mbox{\mathbf$\scriptscriptstyle#1$}}}

\DeclareFontFamily{U}{mathx}{\hyphenchar\font45}
\DeclareFontShape{U}{mathx}{m}{n}{<-> mathx10}{}
\DeclareSymbolFont{mathx}{U}{mathx}{m}{n}
\DeclareMathAccent{\widebar}{0}{mathx}{"73}

\usepackage{soulutf8} \soulregister\cite7 \soulregister\ref7
\soulregister\pageref7 \usepackage{hyperref}
\usepackage[color=yellow]{todonotes} \hypersetup{final} \usepackage{mathrsfs}
\usepackage[advantage,asymptotics,adversary,sets,keys,ff,lambda,primitives,events,operators,probability,logic,mm,complexity]{cryptocode}

\usepackage{cite} 
\usepackage{booktabs}
\usepackage{paralist}
\usepackage[innerleftmargin=5pt,innerrightmargin=5pt]{mdframed}
\usepackage{caption}
\captionsetup{belowskip=0pt}
\usepackage{bm}
\usepackage{url}
%\usepackage{dirtytalk}
\newcommand{\say}[1]{\emph{``#1''}}
\usepackage[margin=0.7in,a4paper]{geometry}
\usepackage[normalem]{ulem}
\usepackage{dashbox}
\newcommand{\dboxed[1]}{\dbox{\ensuremath{#1}}}
\usepackage{hyperref}
\usepackage[capitalise]{cleveref}

%\usepackage{mathptmx}

\mathchardef\mhyphen="2D

\newcommand{\newm}[1]{{\textcolor{red}{#1}}}
%%%%% General commands %%%%%
\newcommand{\variable}[1]{\mathsf{#1}}
\newcommand{\constant}[1]{\mathtt{#1}}
%\newcommand{\comment}[1]{}
%%%%% Complexity classes %%%%%
\renewcommand{\pccomplexitystyle}[1]{\ensuremath{\mathsf{#1}}}
\newcommand{\IP}{\pccomplexitystyle{IP}}
\newcommand{\PSPACE}{\pccomplexitystyle{PSPACE}}

%%%%% Projects %%%%%
\newcommand{\project}[1]{\ensuremath{\mathtt{#1}}}
\newcommand{\zeth}{\project{ZETH}}
\newcommand{\zecale}{\project{Zecale}}
\newcommand{\zexe}{\project{ZEXE}}
\newcommand{\ethereum}{\project{Ethereum}}
\newcommand{\coda}{\project{Coda}}
\newcommand{\nym}{\project{Nym}}
\newcommand{\tor}{\project{TOR}}
\newcommand{\loopix}{\project{Loopix}}
\newcommand{\dizk}{\project{DIZK}}
\newcommand{\libsnark}{\project{libsnark}}
\newcommand{\libff}{\project{libff}}

%%%% Predicates
\newcommand{\zecaleP}{\algostyle{ZecaleP}} % Zecale Predicate
\newcommand{\appP}{\algostyle{BaseAppP}} % BaseApp Predicate
\newcommand{\predicate}{\algostyle{P}} % general predicate

%%%%% Assets %%%%%
\newcommand{\asset}[1]{\mathtt{#1}}
\newcommand{\ether}{\asset{Ether}}

% %%%%% Party %%%%%
% \newcommand{\party}[1]{\mathit{#1}}
% %\newcommand{\prover}{\ensuremath{\party{P}}\xspace}
% %\newcommand{\verifier}{\ensuremath{\party{V}}\xspace}
% \newcommand{\miner}{\ensuremath{\party{Miner}}\xspace}
% \newcommand{\challenger}{\party{C}}

%%%%% General crypto commands %%%%%
%\let\vec\bm%
\newcommand{\smallset}[1] {\{#1\}}
\newcommand{\fset}[1] {\left\{ #1 \right\}}
\newcommand{\params}{\variable{pp}} % public parameters
\newcommand{\keyspace}{\funspace{K}}

\newcommand{\family}[1]{\mathcal{#1}}

\newcommand{\ENC}{\pcalgostyle{E}} %encryption scheme

\newcommand{\nuppt}{\pcmachinemodelstyle{NUPPT}} % Non-Uniform PPT

%%%%% Ethereum accounts %%%%%
\newcommand{\accountstyle}[1]{\mathbf{#1}}
%% Smart-contracts accounts
\newcommand{\contractstyle}[1]{\widetilde{\accountstyle{#1}}}
\newcommand{\baseAppContract}{\ensuremath{\contractstyle{baseAppC}}}
\newcommand{\modifiedBaseAppContract}{\ensuremath{\contractstyle{ZbaseAppC}}}
\newcommand{\ZecaleContract}{\ensuremath{\contractstyle{ZecaleC}}}

%%%% Variables %%%%
%\newcommand{\ledger}{\mathscr{L}}

%%%% Zecale commands %%%%%
\newcommand{\zktx}{\variable{zktx}} % transaction containing a SNARK
\newcommand{\aggrtx}{\variable{aggrtx}} % Aggregation tx containing the wrapping SNARK
\newcommand{\inputs}{\variable{inputs}} % Application inputs (e.g. Zeth inputs)

\newcommand{\execAppLogic}{\algostyle{ExecAppLogic}} % Application logic (eg. Zeth state transition logic) executed if the SNARK verifies correctly
%
\newcommand{\zkAppCRS}{\variable{\crs_{app}}} % CRS of the base application (e.g. Zeth)
\newcommand{\zkOtherAppCRS}{\variable{\crs_{\widetilde{app}}}}
\newcommand{\zethCRS}{\variable{\crs_{zeth}}} % CRS of Zeth
\newcommand{\zecaleCRS}{\variable{\crs_{zec}}}
\newcommand{\zecaleRelation}{\algostyle{ZecaleRelation}}
%
\newcommand{\zkpZecale}{\variable{\pi_{zec}}}
\newcommand{\zkpBaseApp}{\variable{\pi_{app}}}
\newcommand{\zkpBaseOtherApp}{\variable{\pi_{\widetilde{app}}}}
%
\newcommand{\inpZecale}{\variable{\inp_{zec}}}
\newcommand{\inpBaseApp}{\variable{\inp_{app}}}
\newcommand{\inpBaseOtherApp}{\variable{\inp_{\widetilde{app}}}}
%
\newcommand{\ledgerZecale}{\ledger_{zec}}
\newcommand{\ledgerApp}{\ledger_{app}}
%
\newcommand{\snarkZecale}{\algostyle{\snark_{zec}}}
\newcommand{\snarkApp}{\algostyle{\snark_{app}}}
\newcommand{\snarkOtherApp}{\algostyle{\snark_{\widetilde{app}}}}
%
\newcommand{\appContractAddress}{\variable{baseAppAddr}}
\newcommand{\ZappContractAddress}{\variable{ZbaseAppAddr}}
\newcommand{\zecaleContractAddress}{\variable{zecaleAddr}}
\newcommand{\dispatchData}{\variable{dispatchData}}
\newcommand{\dispatch}{\algostyle{dispatch}}
\newcommand{\createContractInstance}{\algostyle{createContractInstance}}
%
\newcommand{\gasSaved}{\variable{gSaved}}
%
\newcommand{\snarkBatch}{\algostyle{VBATCH}} % Verify batch - equation
% SNARK verifier equation
\newcommand{\verifierEq}{\algostyle{\verifier_{eq}}}

%% Commands used in protocols and procedures
\newcommand{\inpH}{\variable{h}}
\newcommand{\inpPacked}{\variable{xH}}
\newcommand{\inpValidity}{\variable{xValid}}
\newcommand{\processAggregatedTx}{\algostyle{processAggrTx}}
\newcommand{\processTx}{\algostyle{processTx}}
\newcommand{\constructor}{\algostyle{constructor}}
% Map representing the storage memory of the contract
\newcommand{\storage}{\variable{storage}}
%
\newcommand{\vkHash}{\variable{vkHash}}
%
\newcommand{\broadcastMsg}{\algostyle{broadcastMsg}}
\newcommand{\encodeToBytes}{\algostyle{encodeToBytes}}
\newcommand{\byteData}{\variable{byteData}}
% Message sender solidity util
\newcommand{\msgSender}{\variable{msgSender}}
%
\newcommand{\decodeBytes}{\algostyle{decodeBytes}}
\newcommand{\decodedData}{\variable{decodedData}}
%
\newcommand{\fieldAppCRS}{\variable{``app\_crs"}}
\newcommand{\fieldZecaleAddr}{\variable{``zecale\_addr"}}
%
\newcommand{\toDigest}{\algostyle{toDigest}}
\newcommand{\toField}{\algostyle{toField}}
%
\newcommand{\rNested}{\variable{r_n}}
\newcommand{\rWrapping}{\variable{r_w}}


%%%%% Constants %%%%%
\newcommand{\batchSize}{\constant{BATCH\_SIZE}} % Size of the aggregation batch
\newcommand{\gasPrice}{\constant{gPrice}} % Gas price on Ethereum
\newcommand{\blockReward}{\constant{BLOCK\_REWARD}} % Block reward
\newcommand{\aggrReward}{\constant{AGGREGATION\_REWARD}} % Aggregation reward
\newcommand{\blockGasLimit}{\constant{BLOCK\_GAS\_LIMIT}} % Block gas limit
\newcommand{\aggrTxGas}{\constant{AggrTxGas}} % Gas required to pay to settle an aggregation transaction on-chain
\newcommand{\aggrTxCost}{\constant{AggrTxCost}} % Cost of settling an aggregation tx on-chain = gas * gasPrice
\newcommand{\appTxGas}{\constant{AppTxGas}} % Gas required to pay to settle an app transaction on-chain
\newcommand{\appTxCost}{\constant{AppTxCost}} % Cost of settling an app tx on-chain = gas * gasPrice
\newcommand{\verifProofGas}{\constant{VProofGas}} % Gas required to verify a SNARK on-chain
\newcommand{\verifNProofGas}{\constant{VNProofGas}} % Gas to verify Nested proof
\newcommand{\verifWProofGas}{\constant{VWProofGas}} % Gas to verify Wrapping proof
\newcommand{\verifProofCost}{\constant{VProofCost}} % Cost required to verify a SNARK on-chain
\newcommand{\txDefaultGas}{\constant{DGAS}}

%%%%% Cryptographic experiments %%%%%
\newcommand{\game}[1]{\pcalgostyle{#1}}
\newcommand{\ngame}[2]{\pcalgostyle{Game_{#1}^{#2}}}
\newcommand{\parcase}[1]{\noindent\textit{{#1}.}}
\newcommand{\aggrInd}{\game{AGGR\mhyphen IND}} % Aggregation IND game
\newcommand{\zclSND}{\game{ZCL\mhyphen SND}} % Soundness game
\newcommand{\collRes}{\game{coll\mhyphen res}} % Collision res
\newcommand{\snarkSND}{\game{SNARK\mhyphen SND}} % SNARK SDN game

%%%%% Curves %%%%%
\newcommand{\curve}[1]{\mathsf{#1}}
\newcommand{\BNCurve}{\curve{BN\mhyphen{}254}}
\newcommand{\BLSZcash}{\curve{BLS12\mhyphen{}381}}
\newcommand{\BLSZexe}{\curve{BLS12\mhyphen{}377}}
\newcommand{\BWSix}{\curve{BW6\mhyphen{}761}}
\newcommand{\CPcurve}{\curve{CP}}
% May need to distinguish between MNT{4,6}-298 and MNT{4,6}-763, see how it goes, and if I use both
\newcommand{\MNTFour}{\curve{MNT4}}
\newcommand{\MNTSix}{\curve{MNT6}}

%%%%% Algorithms/Schemes %%%%%
\newcommand{\algostyle}[1]{\pcalgostyle{#1}}
\newcommand{\oracle}[1]{\algostyle{O}^{#1}} % Oracle
\newcommand{\produceBlock}{\algostyle{ProdBlock}}
\newcommand{\setup}{\algostyle{Setup}}
\newcommand{\includeTxInBlock}{\algostyle{IncludeTxInBlock}}
% Zeth commands
\newcommand{\mix}{\algostyle{Mix}}
\DeclareMathOperator\shr{shr}% shift right
\DeclareMathOperator\IM{Im}

%%%%% SNARKs notations %%%%%
\newcommand{\snark}{\algostyle{\Psi}}
\newcommand{\LAN}{\mathbf{L}}
\newcommand{\LANZECALE}{\LAN^{\project{zec}}}
\newcommand{\pcrelstyle}[1]{\mathbf{#1}}
\newcommand{\REL}{\pcrelstyle{R}}
\newcommand{\LANG}{\pcrelstyle{L}}
\newcommand{\RELZECALE}{\REL^{\project{zec}}}
\newcommand{\RELGEN}{\mathcal{R}}
\newcommand{\RELCIRC}{\REL^{\project{z}}}
\newcommand{\transcript}{\mathsf{Transcript}}
%
\newcommand{\groth}{\algostyle{Groth16}}
\newcommand{\zkproof}{\variable{\pi}} % a zk proof
\newcommand{\zkproofsim}{{\zkproof_{\simulator}}}
\newcommand{\piA}{\variable{A}} % A element of the proof
\newcommand{\piB}{\variable{B}} % B element of the proof
\newcommand{\piC}{\variable{C}} % C element of the proof
\newcommand{\polU}{\variable{u}} % Polynomial U
\newcommand{\polV}{\variable{v}} % Polynomial V
\newcommand{\polW}{\variable{w}} % Polynomial W
\newcommand{\vkABC}{\variable{vkABC}} % Set of ABC ratio elements in the VK
\newcommand{\crs}{\variable{crs}} % common reference string
\newcommand{\master}{\variable{mstr}}
\newcommand{\specialized}{\variable{spec}}
\newcommand{\crsmaster}{\crs^{\master}}
\newcommand{\crsspec}{\crs^{\specialized}}
\newcommand{\srs}{\variable{srs}} % structured reference string
\newcommand{\td}{\variable{td}} % trapdoor for srs
\newcommand{\inp}{\variable{x}} % input
\newcommand{\wit}{\variable{w}} % witness
\newcommand{\proofsystem}{\pcalgostyle{\Psi}}


%%%%% Math/Algebra notations %%%%%%
\newcommand{\paramGen}{\mathcal{G}}
\newcommand{\suchthat}{\text{s.t.}} % Helper for "such that"
\newcommand{\GRP}{\mathbb{G}} % Groupb
\newcommand{\GRPord}{r} % Group order
\newcommand{\Id}{\mathcal{O}} % Point at infinity. Notation for the identity element of our groups
\newcommand{\pair}{e} % Pairing
%\newcommand{\pair}[2]{e(#1, #2)} % Pairing
\newcommand{\ggen}{\mathfrak{g}} % Generator g
\newcommand{\hgen}{\mathfrak{h}} % Generator h
\newcommand{\isEq}{\stackrel{?}{=}}
\newcommand*{\QED}{\hfill\ensuremath{\square}} % QED symbol
\newcommand{\cardinality}[1]{|#1|}
%
\newcommand{\vectorspace}[1]{\mathcal{#1}} % Notation for vector space
\newcommand{\lintransform}[1]{\mathcal{#1}} % Notation for linear transformation
% Operator for the Kernel of the linear transformation
\DeclareMathOperator{\Ker}{Ker}

\newcommand{\eps}{\varepsilon}

%%general cryptography
\newcommand{\fail}{\event{fail}}

%DGKOS-specific notation
\newcommand{\AL}{\variable{AL}}
\newcommand{\flag}[1]{\variable{#1}}
\newcommand{\ready}{\flag{ready}}


%%%% Missing general crypto %%%%%%
\newcommand{\range}[2]{[#1 .. #2]}
\newcommand{\ro}{\mathcal{H}} %random oracle
\newcommand{\hashf}{\mathsf{H}} % hash function 
\newcommand{\rand}[1]{\mathsf{Rnd}(#1)}
\newcommand{\func}[1]{\mathbf{F}_{#1}}
\newcommand{\prot}[1]{\mathbf{P}_{#1}}
\renewcommand{\st}[1]{\uppercase{\texttt{#1}}}
\newcommand{\listvar}[1]{{\variable{list}_{#1}}}

%% FOR UC
\newcommand{\sgen}{\pcalgostyle{SGen}}
\newcommand{\msg}[1]{\mathtt{#1}}

%%%%% Roles %%%%%
%\renewcommand{\pcadvstyle}[1]{\pcalgostyle{#1}}
\newcommand{\role}[1]{\pcadvstyle{#1}}
\newcommand{\user}{\role{U}}
\newcommand{\noofu}{{\noofparties}}
\newcommand{\reg}{\role{R}}
\newcommand{\owner}{\role{O}}
\newcommand{\account}{\role{T}}
\newcommand{\identity}{\role{I}}
\newcommand{\party}{\role{P}}
\newcommand{\pid}{\variable{pid}}
\newcommand{\sid}{\variable{sid}}
\newcommand{\polas}{\role{R}}
\newcommand{\noofpolas}{\variable{r}}
\newcommand{\idprov}{\role{I}}
\newcommand{\noofidprov}{\variable{l}}

\newcommand{\pccom}[1]{{\footnotesize{/\hspace*{-2pt}/ #1}}\newline}

%%%%% For compliance %%%%%%
\newcommand{\prfkey}{\pcvarstyle{k}}
\newcommand{\policydig}{\pcvarstyle{pd}}

\newcommand{\tx}{\variable{tx}} %transaction
\newcommand{\comp}{\variable{comp}} %index for compliance variables

%%%% For compliance relation graph
\newcommand{\pcvarstyle}[1]{\mathsf{#1}} %for variables
\newcommand{\proc}[1]{\mathsf{#1}} % for procedures

\newcommand{\cred}{\variable{cred}}
\newcommand{\pcred}{\variable{pcred}}
\newcommand{\scred}{\variable{scred}}

%%% threshold encryption and anonymity revokers
\newcommand{\ar}{\role{Z}}%{\role{A\hspace*{-0.1em}R}} %anonymity revokers
\newcommand{\thresholdar}{\pcvarstyle{t}}
\newcommand{\noofar}{\pcvarstyle{m}}
\newcommand{\ciphertext}{\pcvarstyle{c}}
\newcommand{\ckey}{\ciphertext_{key}}
\newcommand{\cuser}{\ciphertext_{pcred}}
\newcommand{\cid}{\ciphertext_{id}}
\newcommand{\cpol}{\ciphertext_{pol}}

\newcommand{\combine}{\pcalgostyle{Comb}}
\newcommand{\TENC}{\pcalgostyle{TE}}
\newcommand{\BSIG}{\pcalgostyle{BS}}
\newcommand{\unblind}{\pcalgostyle{Unblind}}
\newcommand{\threshold}{\pcvarstyle{t}}
\newcommand{\noofparties}{\pcvarstyle{n}}
\newcommand{\varshare}{\pcvarstyle{s}}
\newcommand{\simpart}{\pcalgostyle{SimPart}}

\renewcommand{\SS}{\pcalgostyle{SS}}
\newcommand{\share}{\pcalgostyle{Share}}
\newcommand{\reconstruct}{\pcalgostyle{Reconstruct}}

\newcommand{\app}{\mathbf{A}}
\newcommand{\maxacc}{\variable{maxAccounts}}
\newcommand{\maxtime}{\variable{time}}
\newcommand{\cnt}{\variable{Count}}
\newcommand{\generateAccount}{\pcalgostyle{generateAccount}}

\newcommand{\RELpol}{\REL_{pol}}
\newcommand{\RELacc}{\REL_{acc}}
\newcommand{\RELval}{\REL_{val}}
\newcommand{\RELcomp}{\REL_{comp}}

\newcommand{\funcidis}{\func{id\mhyphen is}}
\newcommand{\funccompis}{\func{comp\mhyphen is}}

\newcommand{\proofsystemc}{{\proofsystem_\texttt{comp}}} % proof system for compliance
\newcommand{\proofc}{{\zkproof_{\texttt{comp}}}} %proof of compliance
\newcommand{\proofacc}{\zkproof_{\texttt{acc}}}
\newcommand{\zkproofsk}{\zkproof_{\sk}}

% modules
\newcommand{\module}[1]{\mathscr{#1}}
\newcommand{\transmodule}{\module{T}}
\newcommand{\identitymodule}{\module{I}}
\newcommand{\compliancemodule}{\module{C}}
\newcommand{\ledgermodule}{\module{L}}
\newcommand{\ledger}{\ledgermodule}
\newcommand{\regulator}{\module{R}}

\newcommand{\aux}{\pcvarstyle{aux}}
\newcommand{\saux}{\pcvarstyle{saux}}
\newcommand{\paux}{\pcvarstyle{paux}}
\newcommand{\auxid}{\aux_{id}}
\newcommand{\auxpol}{\aux_{pol}}
\newcommand{\auxacc}{\aux_{acc}}
\newcommand{\pkacc}{\pk_{acc}}
\newcommand{\skacc}{\sk_{acc}}
\newcommand{\auxtx}{\aux_{tx}}
\newcommand{\sauxtx}{\saux_{tx}}
\newcommand{\pauxtx}{\paux_{tx}}

\newcommand{\idis}{id\mhyphen is}

\newcommand{\policy}[1]{\mathbf{#1}}
\newcommand{\comppolicy}{\policy{P}} % compliance policy
\newcommand{\valpolicy}{\policy{V}} % transaction validity policy
\newcommand{\revpolicy}{\policy{R}} % policy that determines which user's actions lead to its identity revelation
\newcommand{\rec}[1]{\pckeystyle{#1}}
\newcommand{\recpolicy}{\rec{policy}}
\newcommand{\recpolicydig}{\rec{policyDigest}}

\newcommand{\prooff}{\psi_\texttt{fraud}} % proof of fraud

\newcommand{\prep}{\pcvarstyle{p}}
\newcommand{\prepcomp}{{{\proofsystemc.\prep}}}

\newcommand{\desc}[1]{\noindent\underbar{#1:}}

%%% signature
\newcommand{\signaturescheme}{\pcalgostyle{S}}
\newcommand{\SIG}{\signaturescheme}
\newcommand{\signature}{\sigma}
%\newcommand{\pcoraclestyle}[1]{\mathsf{#1}}
\newcommand{\oracleo}{\pcoraclestyle{O}}

\newcommand{\accept}{\pcvarstyle{accept}}

\renewcommand{\prover}{\pcalgostyle{Pro}}
\renewcommand{\verifier}{\pcalgostyle{Ver}}


%%%%% Comments %%%%%
\definecolor{bananamania}{rgb}{0.98,0.91,0.71}
\definecolor{darkred}{rgb}{0.7,0,0}
\definecolor{blueish}{rgb}{0.1,0.1,0.5}
\definecolor{pinkish}{rgb}{0.9,0.8,0.8}

\definecolor{redone}{HTML}{ffe5e0}
\definecolor{yellowone}{HTML}{fffcf2}
\definecolor{pinkone}{HTML}{fff2f2}
\definecolor{blueone}{HTML}{f2f9fd}

\DeclareRobustCommand{\antoines}[2]{{\color{darkred}\sethlcolor{bananamania}\hl{\textbf{Antoine #1:} #2}}}

\DeclareRobustCommand{\michals}[2]{{\color{blueish}\sethlcolor{pinkish}\hl{\textbf{Michal #1:} #2}}}


\title{On PIOP-based zkSNARKs}

\author{} 
%\iflncs{
\institute{} 

\allowdisplaybreaks

\begin{document} \sloppy \maketitle

\begin{abstract}
  In this paper we investigate properties of zkSNARKs obtained by
  compiling a PIOP proof using a polynomial commitment scheme. The question we
  try to answer is \say{What polynomial commitment's properties propagate to
    the resulting zkSNARK?}. The properties we focus on are:
  \begin{compactenum}
  \item simulation extractability,
  \item SRS updatability,
  \item SRS-updatable simulation extractability,
  \item subversion zero knowledge.
  \end{compactenum}
  The research hypothesis is \say{A NIZK obtained from a simulation extractable /
    SRS-updatable / SRS updatable SE / subversion zero knowledge polynomial
    commitment is simulation extractable /
    SRS-updatable / SRS updatable SE / subversion zero knowledge.} To be able to
  show the hypothesis we need to solve a number of problems
  \begin{compactenum}
  \item Neither simulation extractability, SRS-updatability, SRS-updatable
    simulation extractability, nor subversion zero knowledge have been defined
    for a polynomial commitment scheme. Another contribution of the paper is
    defining these properties. 
  \item Similarly, there is no definition for SRS-updatable simulation
    extractable NIZKs.
  \item The polynomial IOP is defined very generally, cf.~\cref{def:piop}, what
    makes showing generic properties difficult. The paper would propose tighter
    definitions which would emphasize more the structure of PIOP,
    but would not narrow a class of possible (from the practical point of view)
    PIOPs too much, cf.~\cref{def:wepiop,def:sdwepiop}.
  \end{compactenum}
  
\end{abstract}

\paragraph{Polynomial commitment scheme.}
\label{sec:poly_com}
In the polynomial commitment scheme $\PCOM = (\kgen, \com, \open, \verify)$ the
prover $\prover$ convinces the verifier $\verifier$ that some polynomial $\p{f}$
which $\prover$ committed to evaluates to $s$ at some point $z$ chosen by
$\verifier$.  The key generation algorithm $\kgen$ takes as input a security
parameter $\secparam$ and a parameter $\maxdeg$ which determines the maximal
degree of the committed polynomial. We assume that $\maxdeg$ can be read from
the output SRS.
  
We emphasize the following properties of a secure polynomial commitment
$\PCOM$:
\begin{description}
\item[Evaluation binding:] A $\ppt$ adversary $\adv$ which outputs a commitment
  $\vec{c}$ and evaluation points $\vec{z}$ has at most negligible chances to
  open the commitment to two different evaluations $\vec{s}, \vec{s'}$. That is,
  let $k \in \NN$ be the number of committed polynomials, $l \in \NN$ number of
  evaluation points, $\vec{c} \in \GRP^k$ be the commitments,
  $\vec{z} \in \FF_p^l$ be the arguments the polynomials are evaluated at,
  $\vec{s},\vec{s}' \in \FF_p^k$ the evaluations, and
  $\vec{o},\vec{o}' \in \FF_p^l$ be the commitment openings. Then for every
  $\ppt$ adversary $\adv$
	\[
		\Pr
			\left[
			\begin{aligned}
				& \verify(\srs, \vec{c}, \vec{z}, \vec{s}, \vec{o}) = 1,  \\ 
				& \verify(\srs, \vec{c}, \vec{z}, \vec{s}', \vec{o}') = 1, \\
				& \vec{s} \neq \vec{s}'
			\end{aligned}
			\,\left|\,\vphantom{\begin{aligned}
                  & \\
                  & \\
                  &
                \end{aligned}}
			\begin{aligned}
				& \srs \gets \kgen(\secparam, \maxdeg),\\
				& (\vec{c}, \vec{z}, \vec{s}, \vec{s}', \vec{o}, \vec{o}') \gets \adv(\srs)
			\end{aligned}
			\right.\right] \leq \negl\,.
	\]

\end{description}
	

\begin{description}
\item[Commitment of knowledge] For every $\ppt$ adversary $\adv$ who produces
  commitment $c$, gets random evaluation point $z$, and outputs evaluation $s$
  with an opening $o$ there exists a $\ppt$ extractor $\ext$ such that
\[
  \Pr \left[
    \begin{aligned}
      & \p{f} = \ext_\adv(\srs, c; r),\\
      & c = \com(\srs, \p{f}),\\
      & \verify(\srs, c, z, s, o) = 1
    \end{aligned}
    \,\left|\,
      \vphantom{
        \begin{aligned}
          & \\
          & \\
          &
        \end{aligned}
        }
    \begin{aligned}
      & \srs \gets \kgen(\secparam, \maxdeg),\\
      & (c, z, s, o) \gets \adv(\srs; r)
    \end{aligned}
  \right.\right]
  \geq 1 - \epsk(\secpar).
\]
In that case we say that $\PCOM$ is $\epsk$-knowledge.
\end{description}
Intuitively when a commitment scheme is a commitment of knowledge then if an
adversary produces a (valid) commitment $c$, which it can open, then it also
knows the underlying polynomial $\p{f}$ which commits to that value.

\begin{definition}[Simulation extractable polynomial commitment]
  \label{def:sepcom}
  Let $\PCOM$ be a polynomial commitment scheme, let $\oracles$ be an 
  oracle which on input
  \begin{description}
%   \item [$(\msg{commit}, f)$:] returns commitment $c = \com(f)$ and adds $(f,
% c)$ to list $Q$.
\item[$(\msg{commit}, f, d)$:] for $\deg(f) = d'$, $d' \leq d \leq \maxdeg$,
  picks $d - d'$ random elements $r_1, \ldots, r_{d - d'}$, sets 
  polynomial $g(X) = f(X) + r_1 X^{d' + 1} + \ldots + r_{d - d'} X^{d}$, returns
  commitment $c = \com(g)$ and adds $(g, c)$ to list $\Qcom$.
  \item[$(\msg{evaluate}, c, z)$:] returns $f(z)$ where $f$ is a polynomial
    which commitment is $c$ and $(f, c) \in \Qcom$; add $z$ to $\Qev$.
  \item[$(\msg{open}, c, x, y)$:] returns an opening $o$ for commitment $c$
    assuring that for some polynomial $f$, such that $c \in \image(\com(f))$,
    holds $f(x) = y$.
  \end{description}
  We say that $\PCOM$ is \emph{simulation extractable} if for any $\ppt$
  adversary $\adv$ with oracle access to $\oracles$ there exists extractor
  $\ext$ such that
\[
  \Pr \left[
    \begin{aligned}
      & \p{f} = \ext_\adv(\srs, c; r),\\
      & c = \com(\srs, \p{f}),\\
      & \verify(\srs, c, z, s, o) = 1,\\
      & z \not\in \Qev
    \end{aligned}
    \,\left|\,
      \vphantom{
        \begin{aligned}
          & \\
          & \\
          & \\
          &
        \end{aligned}
        }
    \begin{aligned}
      & \srs \gets \kgen(\secparam, \maxdeg),\\
      & (c, z, s, o) \gets \adv^{\oracles}(\srs; r), \\
      % & z \sample \FF,\\
      % & (s, o) \gets \adv^{\oracles}(\srs, z; r)
    \end{aligned}
  \right.\right]
  \geq 1 - \epsk(\secpar).
\]
%\michals{21.06}{Write it as a game.}
  \michals{23.04}{Can $\ext$ ask $\adv$ to give evaluations of the committed
    polynomial? That is how $\ext$ in a proof system works -- it evaluates
    polynomials submitted by the adversary on multiple challenges.}
 \end{definition}

 \begin{definition}[Rational SE PCOM]
   \label{def:rational_sepcom}
   We call a simulation extractable polynomial commitment $\PCOM$
   \emph{rational} if $\oracles$ on input $(\msg{commit}, f, d)$ accepts $f$
   being a rational function (not only a polynomial).
 \end{definition}

 \begin{definition}[$f$-SE PCOM]
   \label{def:f_sepcom}
   We call a simulation extractable polynomial commitment $\PCOM$ $f$-simulation
   extractable if $\oracles$ accepts input
   $(\msg{operate}, f, c_1, \ldots, c_k)$. On that input, $\oracles$ retrieves
   polynomials $f_1, \ldots, f_k$ which are committed in $c_1, \ldots, c_k$, 
   returns $c = \com(f(f_1, \ldots, f_k))$, and adds $(f(f_1, \ldots, f_k), c)$
   to $\Qcom$.
 \end{definition}

 \begin{definition}[\hl{Good name needed}]
   We call a commitment scheme $\PCOM$ \hl{good name needed} if any $\adv$ who
   outputs commitments $c_1, \ldots, c_k$, evaluation point $z$, evaluations
   $s_1, \ldots, s_k$ and batch opening $o$, which certifies that polynomials
   $f_i$ evaluates at $z$ to $s_i$ and $c_i \in \image(\com(f_i))$, can also
   produce valid openings $o_i$ for each triple $(c_i, z, s_i)$.
 \end{definition}
 \michals{23.06}{That's easy for KZG batched as in Plonk (with minor difference)
 -- just get a number of batch openings and do Gaussian elimination.}

 \begin{definition}[Sufficiently simulatable PCOM]
   Let $\PCOM$ be a SE polynomial commitment scheme and
   $\PS = (\kgen, \prover, \verfier, \simulator)$ a zero-knowledge proof system
   for relation $\REL$. We call $\PCOM$ \emph{sufficiently simulatable for
     $\PS$} if there exist a $\ppt$ algorithm $\adv$ such that for
   $\srs \sample \kgen$, all $(\inp, \wit) \in \REL$ holds
   \[
     \SD_{\srs \sample \kgen(\secparam)}(\prover(\srs, \inp, \wit),
     \adv^{\oracles}(\srs, \inp)).
   \]
 \end{definition}
 Intuitively, we call $\PCOM$ sufficiently simulatable for $\PS$ if a $\ppt$
 $\adv$ given access to $\PCOM$'s simulator oracle $\oracles$ can produce a
 simulated proof for $\PS$.

 \begin{definition}[Polynomial IOP,~\cite{EPRINT:Szepieniec20}]
  \label{def:piop}
  Let $\REL$ be an indexed relation with a corresponding language $\LANG$, $\FF$
  some finite field, and $\maxdeg$ a degree bound and $\noofp$ a parameter. A
  \emph{polynomial IOP for $\REL$ with degree bound $\maxdeg$} is a pair of
  interactive machines $\prover, \verifier$ such that
\begin{itemize}
\item $(\prover, \verifier)$ is an interactive proof for $\LANG$ with $r$ rounds
  and soundness error $\epss$;
\item $\prover$ sends to $\verifier$ polynomials $f_i \in \FF[X]$,
  $i \in \range{1}{\noofp}$, of degree at most $\maxdeg$;
\item $\verifier$ is an oracle machine with access to a list of oracles, which
  contains one oracle for each polynomial it has received from the prover;
\item When an oracle associated with a polynomial $f_i(X)$ is queried on a point
  $z_j \in \FF$, the oracle responds with the value $f_i(z_j)$; 
\item $\verifier$ sends challenges $\alpha_k \in \FF$ to $\prover$;
\item $\verifier$ is public coin.
\end{itemize}
\end{definition}

\michals{28.04}{Add preprocessing and zero knowledge}

\begin{definition}[Witness encoding PIOP (WEPIOP)]
  \label{def:wepiop}
  Let $\PS$ be a PIOP.  We say that $\PS$ is \emph{witness encoding} if there is
  a function $\decode$ and set $\encset \in [\noofp]$ such that for any
  $(\inp, \wit) \in \REL$ and polynomials $\smallset{f_i}_{i \in [\noofp]}$ sent by an
  honest prover, $\decode(\smallset{f_i}_{i \in \encset}) = \wit$. We call $\encset$ the
  \emph{encoding set}.
\end{definition}
In other words, PIOP is witness encoding if for any valid proof for a statement
$\inp$ in the language, the corresponding witness can be read from the
polynomial coefficients. We note that this is the case for virtually all
PIOPs. \michals{28.04}{Check!}]

\begin{definition}[Somehow deterministic WEPIOP]
  \label{def:sdwepiop}
  Let $\PS$ be a WEPIOP for $\REL$. For each polynomial $f_i$ sent by the prover
  denote by $A_i$ the set of challenges sent by the verifier and by $F_i$ the
  set of polynomials sent by the prover \emph{before} the prover sends
  $f_i$. Let $\encset$ be an encoding set. We say that $\PS$ is \emph{somehow
    deterministic} (SD) if for any $(\inp, \wit) \in \REL$, polynomials
  $\smallset{f_i}_{i \in [\noofp]}$ send by the prover, and encoding set
  $\encset \neq [\noofp]$ each polynomial
  $f_j \in \smallset{f_i}_{i \in [\noofp] \setminus \encset}$ is determined by
  \begin{itemize}
    \item polynomials $F_j$, and
    \item the verifier's challenges $A_i$, and
    \item the witness $\wit$.
  \end{itemize}
\end{definition}
Intuitively, we say that WEPIOP is somehow deterministic if the only
non-deterministic messages send by the prover are polynomials encoding the
witness, and all other messages are determined by the previous one, witness, and
verifier's challenges.

\begin{lemma}
  Let $\PS = (\PS.\prover, \PS.\verifier)$ be a PIOP with knowledge soundness error
  $\epsks$ where the prover sends up to $\noofp$ polynomials. Let $\PCOM$ be a
  knowledge polynomial commitment scheme with extraction error $\epsext$. Let
  $\PSc = (\PSc.\prover, \PSc.\verifier)$ be a proof system such that
  \begin{description}
  \item[$\PSc.\prover$] acts as $\PS.\prover$, except
    \begin{itemize}
    \item when $\PS.\prover$ sets up a polynomial oracle $\oracleo_f$,
      $\PSc.\prover$ sends commitment $c = \com(f)$;
    \item when $\PSc.\verifier$ asks $\PSc.\prover$ to open a commitment
      $c = \com(f)$ at $z$ it returns $f(z)$ and a proof $o$ of correctness of
      the opening.
  \end{itemize}
  \item[$\PSc.\verifier$] acts as $\PS.\verifier$, except when $\PS.\verifier$
    asks oracle $\oracleo_f$ for an evaluation of $f$ at $z$, $\PSc.\verifier$
    sends $z$ to $\PSc.\prover$ and expects $f(z)$ in return.
  \end{description}
  Then $\PSc$ is knowledge sound with knowledge error at most $\epsks + \noofp \cdot \epsext$.
\end{lemma}
\begin{proof}
  Let $\adv$ be an adversary that breaks knowledge soundness of $\PSc$ with
  probability $\eps$ greater than $\epsks + \noofp \cdot \epsext$.  We build
  $\bdv$ that breaks knowledge soundness of $\PS$ with probability greater than
  $\epsks$. More precisely, let $\ext_\adv$ be an extractor for $\adv$ and
  $\ext_\bdv$ be an extractor for $\bdv$. We show that if $\adv$ produces an
  instance and proof such that $\ext_\adv$ fails to extract the corresponding
  witness, then $\bdv$ produces an instance and proof that $\ext_\bdv$ fails as well.

  $\bdv$ starts by picking an SRS $\srs$ for the polynomial commitment scheme
  $\PCOM$ and provides $\srs$ to $\adv$. Let $\ext^{\PCOM}_\adv$ be a polynomial
  commitment extractor for $\adv$ that for any commitment $c$ produced by $\adv$
  outputs polynomial $f$ such that $c \in \image(\com(f))$, except with
  probability $\epsext$.  Let $\inp$ be the instance $\adv$ creates a proof
  for. Each time $\adv$ sends a commitment $c$, $\bdv$ uses $\ext^{\PCOM}_\adv$
  to extract polynomial $f$ from $c$ and sends $f$ to a newly set oracle
  $\oracleo_f$.  \michals{5.5}{how much we can generalize here?  I.e.~could we
    have a single proof that would cover algebraic adversaries that returns with
    $c$ also a vector of group elements' logarithms; or commitment schemes where
    there is an extraction key $k$ allowing to extract $f$ from $c$?}  Each time
  $\PS.\verifier$ sends a challenge $\alpha$, $\bdv$ passes $\alpha$ to $\adv$
  as $\PSc.\verifier$'s challenge. Note that $\PS.\verifier$ expects an oracle
  for some (challenge-dependent) polynomial $f$ as an answer, thus $\adv$
  replies with a commitment to some polynomial $f'$. $\bdv$ extracts $f'$ and
  sets a new oracle for it.  When $\PS.\verifier$ queries some oracle
  $\oracleo_f$ at $z$ to learn $f(z)$, $\bdv$ replies with $f(z)$, sends $z$
  to $\adv$ and gets $y$. If $\adv$ provides $y \neq f(z)$, then $\bdv$ aborts
  the proof. Importantly, that may happen with probability $\epsev$ only as an
  acceptable proof from $\adv$ would require breaking evaluation binding
  property of $\PCOM$. 
  
  Eventually, $\adv$ finalizes its proof for $\inp$. Note that
  if $\PSc.\verifier$ accepts the proof provided by $\adv$, then $\PS.\verifier$
  accepts the proof provided by $\bdv$.

  Probability that $\bdv$ \emph{fails} is upper-bounded by three events. One is
  that $\adv$ fails to provide a convincing proof such that $\ext_\adv$ does not
  output a corresponding witness. Probability of this event is upper-bounded by
  $(1 - \eps)$. Another is that $\adv$ breaks evaluation binding property of the
  commitment scheme, since $\verifier$ makes up to $\noofev$ evaluations, that
  happens with probability at most $\noofev \cdot \epsev$ Furthermore, $\bdv$
  may fail to extract some of the polynomials which $\adv$ made commitments
  for. Since $\adv$ commits to $\noofp$ polynomials and $\bdv$ fails to extract
  each of them with probability at most $\epsext$, by the union bound $\bdv$
  fails to extract \emph{some} of the polynomials with probability upper-bounded
  by $\noofp \cdot \epsext$.

  Note that if $\adv$ manages to output an instance $\inp$ and proof $\zkproof$
  that extractor $\ext_\adv$ fails to produce a corresponding witness $\wit$,
  then $\ext_\bdv$ also fails to produce a witness for the instance output by
  $\bdv$. This comes from the simple observation, if $\ext_\bdv$ was successful,
  then $\ext_\adv$ could run $\bdv$ internally along with $\adv$ until the former
  outputs instance $\inp$ and proof $\zkproof'$, then use $\ext_\bdv$ to extract
  witness $\wit$ for $\inp$. Importantly, both $\adv$ and $\bdv$ output the same instance.

  Hence the probability of $\bdv$'s failure in generating a proof for instance
  $\inp$ that $\ext_\bdv$ fails to extract a correct witness is upper-bounded by
  \[
    (1 - \eps) + \noofp \cdot \epsext.
  \]
  Hence, for $\eps > \epsks + \noofp \cdot \epsext$ adversary $\bdv$ wins with probability at least
  \[
    \epsks + \noofp \cdot \epsext - \noofp \cdot \epsext  = \epsks.
  \]
  \qed
\end{proof}

\bibliographystyle{alpha}
\bibliography{cryptobib/abbrev1,cryptobib/crypto}

\end{document}