% !TeX spellcheck = en_UK
% \let\accentvec\vec              
\documentclass[runningheads,11pt]{llncs}
%\let\spvec\vec \let\vec\accentvec

\usepackage{amssymb,amsmath}
%\let\vec\spvec
%\usepackage{lmodern}

\usepackage[T1]{fontenc}
% \def\vec#1{\mathchoice{\mbox{\mathbf$\displaystyle#1$}}
% {\mbox{\mathbf$\textstyle#1$}} {\mbox{\mathbf$\scriptstyle#1$}}
% {\mbox{\mathbf$\scriptscriptstyle#1$}}}

\DeclareFontFamily{U}{mathx}{\hyphenchar\font45}
\DeclareFontShape{U}{mathx}{m}{n}{<-> mathx10}{}
\DeclareSymbolFont{mathx}{U}{mathx}{m}{n}
\DeclareMathAccent{\widebar}{0}{mathx}{"73}

\usepackage{soulutf8} \soulregister\cite7 \soulregister\ref7
\soulregister\pageref7 \usepackage{hyperref}
\usepackage[color=yellow]{todonotes} \hypersetup{final} \usepackage{mathrsfs}
\usepackage[advantage,asymptotics,adversary,sets,keys,ff,lambda,primitives,events,operators,probability,logic,mm,complexity]{cryptocode}
%\pcbodylinesep=0.15\baselineskip % MK: got an undefined control sequence error


\usepackage{cite} 
\usepackage{booktabs}
\usepackage{paralist}
\usepackage[innerleftmargin=5pt,innerrightmargin=5pt]{mdframed}
%\usepackage{setspace}
\usepackage{caption}
\captionsetup{belowskip=0pt}
\usepackage{bm}
\usepackage{url}
\usepackage{dirtytalk}
\usepackage[margin=0.7in,a4paper]{geometry}
\usepackage[normalem]{ulem}
\usepackage{dashbox}
\newcommand\dboxed[1]{\dbox{\ensuremath{#1}}}

\usepackage{mathptmx}

\newcommand{\newdefs}[1] {\setlength{\fboxsep}{1pt}\colorbox{gray!20}{\(#1\)}}

\newcommand{\COMMENT}[1]  {}

%general formatting
\newcommand{\pcvarstyle}[1]{\mathsf{#1}}
\newcommand{\comment}[1]{{\color{lightgray}#1}}
\newcommand{\continue}{{\Huge{\hl{$\cdots$}}}}
% \renewcommand{\case}[1]{\medskip\noindent{\fbox{Case #1:}}}
% \newcommand{\ncase}[1]{\medskip\noindent{\fbox{#1:}}}
% \newcommand{\ngame}[1]{\medskip\noindent{\fbox{Game $\game{#1}$:}}}
\newcommand{\mhyph}{\text{-}}
\renewcommand{\case}[1]{\medskip\noindent{\textbf{Case #1:}}}
\newcommand{\ncase}[1]{\medskip\noindent{\textbf{#1:}}}
\newcommand{\ngame}[1]{\medskip\noindent{\textbf{Game $\game{#1}$:}}}
\newcommand{\conclude}{\medskip\noindent{}}
\newcommand{\myskip}{-0.16\baselineskip}

% General mathematics
\newcommand{\range}[2] {[#1 \, .. \, #2]}
\newcommand{\SD}{\Delta}
\newcommand{\smallset}[1] {\{#1\}}
\newcommand{\bigset}[1] {\left\{#1\right\}}
\newcommand{\GRP} {\mathbb{G}}
\newcommand{\HRP} {\mathbb{H}}
\newcommand{\pair} {\hat{e}}
\newcommand{\brak}[1] {\left(#1\right)}
\newcommand{\sbrak}[1] {(#1)}
\newcommand{\alg}[1] {\pcalgostyle{#1}}
\newcommand{\image} {\operatorname{im}}
\newcommand{\myland} {\,\land\,}
\newcommand{\mylor} {\,\lor\,}
\newcommand{\vect}[1] {\operatorname{vect}(#1)}
\newcommand{\w}{\omega}
\newcommand{\const}{\pcpolynomialstyle{const}}
\newcommand{\p}[1]{\pcpolynomialstyle{#1}}
\newcommand{\cp}[1]{\pcalgostyle{c}_{\pcpolynomialstyle{#1}}}
\newcommand{\ptx}{\p{t_{X^0}}}
\newcommand{\ptxsim}{\p{\tilde{t}_{X^0}}}
\newcommand{\ev}[1]{\widetilde{\pcpolynomialstyle{#1}}}
\newcommand{\maxconst}{\pcvarstyle{max}}
\newcommand{\numberofconstrains}{\pcvarstyle{n}}
\newcommand{\noofc}{\numberofconstrains}
\newcommand{\noofw}{\pcvarstyle{m}}
\newcommand{\dconst}{\pcvarstyle{d}}
\newcommand{\multconstr}{\pcvarstyle{n}}
\newcommand{\linconstr}{\pcvarstyle{Q}}
\newcommand{\expected}[1]{\mathbb{E}\left[#1\right]}
\newcommand{\infrac}[2]{#1 / #2}
\newcommand{\eps}{\varepsilon}
\DeclareMathOperator{\SPAN}{Span}
\newcommand{\maxdeg}{\pcvarstyle{maxd}}
%polynomials
\newcommand{\pf}{\p{f}}
\newcommand{\pF}{\p{F}}
\newcommand{\pa}{\p{a}}
\newcommand{\pb}{\p{b}}
\newcommand{\pc}{\p{c}}
\newcommand{\pz}{\p{z}}
\newcommand{\pt}{\p{t}}
\newcommand{\pR}{\p{R}}
\newcommand{\pr}{\p{r}}
\newcommand{\ptlo}{\p{t_{lo}}}
\newcommand{\ptmid}{\p{t_{mid}}}
\newcommand{\pthi}{\p{t_{hi}}}
\newcommand{\pS}{\p{S}}
\newcommand{\pT}{\p{T}}
\newcommand{\pta}{\tilde{\p{a}}}
\newcommand{\ptb}{\tilde{\p{b}}}
\newcommand{\ptc}{\tilde{\p{c}}}
\newcommand{\ptz}{\tilde{\p{z}}}
\newcommand{\ptZH}{\tilde{\p{Z_H}}}
\newcommand{\vX}{\vec{X}}
\newcommand{\vB}{\vec{B}}
\newcommand{\va}{\vec{a}}
\newcommand{\vb}{\vec{b}}
\newcommand{\vc}{\vec{c}}

%reduction
\newcommand{\tb}{\tilde{b}} 
\newcommand{\tw}{\tilde{w}} 

% bilinear maps

\newcommand{\bmap}[2] {\left[#1\right]_{#2}}
\newcommand{\gone}[1] {\bmap{#1}{1}}
\newcommand{\gtwo}[1] {\bmap{#1}{2}}
\newcommand{\gi} {\iota}
\newcommand{\gtar}[1] {\bmap{#1}{T}}
\newcommand{\grpgi}[1] {\bmap{#1}{\gi}}
\newcommand{\gnone}[1]{\left[#1\right]}

\newcommand{\msg}[1]{\mathtt{#1}}

% zero knowledge
\newcommand{\oracleo}{\pcalgostyle{O}}
\newcommand{\crs}{\pcvarstyle{crs}}
\newcommand{\srs}{\pcvarstyle{srs}}
\newcommand{\srss}{\pcvarstyle{srs_{spec}}}
\newcommand{\td}{\pcvarstyle{td}}
\newcommand{\ip}[2]{\left\langle #1, #2\right\rangle}
\newcommand{\zkproof}{\pi}
\newcommand{\proofsystem}{\pcschemestyle{\Psi}}
\newcommand{\PS}{\proofsystem}
\newcommand{\psfs}{\proofsystem_\fs}
% \newcommand{\ps}{\proofsystem}
\newcommand{\nuppt}{\pcmachinemodelstyle{NUPPT}}
\newcommand{\ro}{\mathcal{H}}
\newcommand{\rof}[2]{\mathbf{\Omega}_{#1, #2}}
\newcommand{\trans}{\pcvarstyle{trans}}
\newcommand{\tr}{\pcvarstyle{tr}}
\newcommand{\instsize}{\pcvarstyle{n}}
\newcommand{\KG} {\mathsf{K}}
\newcommand{\kcrs} {\KG_{\crs}}
\renewcommand{\dist}{\ddv}
\newcommand{\fs}{\pcalgostyle{fs}}
\newcommand{\sigmaprot}{\pcalgostyle{\Sigma}}
\newcommand{\se}{\pcvarstyle{se}}
\newcommand{\snd}{\pcvarstyle{snd}}
\newcommand{\zk}{\pcvarstyle{zk}}

\newcommand{\extpcom}{\ext^{\PCOM}}
\newcommand{\extps}{\ext^{\PS}}


%rewinding---tree of transcripts
\newcommand{\pcboolstyle}[1]{\mathtt{#1}}
\newcommand{\treebuild}{\pcalgostyle{TreeBuild}}
\newcommand{\tree}{\pcvarstyle{T}}
\newcommand{\counter}{\pcvarstyle{counter}}


%PLONK related
\newcommand{\pcschemestyle}[1]{\bm{\mathsf{#1}}}
\newcommand{\plonkprot}{\pcschemestyle{P}}
\newcommand{\plonkprotfs}{\pcschemestyle{P}_\fs}
\newcommand{\kzg}{\pcschemestyle{kzg}}
\newcommand{\sonicprot}{\pcschemestyle{S}}
\newcommand{\sonicprotfs}{\pcschemestyle{S}_\fs}
\newcommand{\selector}[1]{\pcvarstyle{q_{#1}}}
\newcommand{\selmulti}{\selector{M}}
\newcommand{\selleft}{\selector{L}}
\newcommand{\selright}{\selector{R}}
\newcommand{\seloutput}{\selector{O}}
\newcommand{\selconst}{\selector{C}}
\newcommand{\chz}{\mathfrak{z}}
\newcommand{\ochz}{{\omega \mathfrak{z}}}
\newcommand{\reduction}{\rdv}
\newcommand{\ch}{\pcvarstyle{ch}}

\newcommand{\game}[1]{\pcalgostyle{G}_{#1}}

\newcommand{\lag}{\p{L}}
\newcommand{\pubinppoly}{\p{PI}}

% general complexity theory
% \newcommand{\RND}[1]{\pcalgostyle{RND}(#1)}
\newcommand{\RND}[1]{\pcvarstyle{R}(#1)}
\newcommand{\RELGEN}{\mathcal{R}}
\newcommand{\REL}{\mathbf{R}}
\newcommand{\LANG}{\mathbf{L}}
\newcommand{\inp}{\pcvarstyle{x}}
\newcommand{\wit}{\pcvarstyle{w}}
\newcommand{\class}[1]{\mathfrak{#1}}
\newcommand{\ig}{\pcalgostyle{IG}}
\newcommand{\accProb}{\event{acc}}
\newcommand{\waccProb}{\event{\widetilde{acc}}}
\newcommand{\frkProb}{\event{frk}}
\newcommand{\extProb}{\event{ext}}
\newcommand{\FS}{\pcalgostyle{fs}} % Fiat-Shamir transform
\newcommand{\aux}{\pcvarstyle{aux}} %auxiliary input
\renewcommand{\state}{\pcvarstyle{st}} %state
% \newcommand{\dlog}{\pcvarstyle{dlog}}
\newcommand{\vereq}{\p{ve}}

\newcommand{\sgen}{\pcalgostyle{SGen}}


\newcommand{\prove}{\pcalgostyle{Prove}}


%Commitment schemes
\newcommand{\COM}{\pcschemestyle{C}}
\newcommand{\PCOM}{\pcschemestyle{PC}}
\newcommand{\PC}{\PCOM}
\newcommand{\srsPC}{\srs_{\PCOM}}
\newcommand{\PCOMp}{\pcschemestyle{PC}_{\plonkprot}}
\newcommand{\PCOMs}{\pcschemestyle{PC}_{\sonicprot}}
\newcommand{\com}{\pcalgostyle{Com}}
\newcommand{\op}{\pcalgostyle{Op}}
\newcommand{\open}{\op}
\newcommand{\oracles}{\oracleo_{\simulator}}
\newcommand{\oraclec}{\oracleo_{\mathsf{chal}}}
\newcommand{\oraclespcom}{{\oracleo_{\simulator}^{\PCOM}}}
\newcommand{\oraclesps}{{\oracleo_{\simulator}^{\PS}}}
\newcommand{\decode}{\pcalgostyle{Decode}}
\newcommand{\van}[1]{v_{#1}}
\newcommand{\noofq}{\pcvarstyle{q}}

\newcommand{\radv}{r_{\adv}}
\newcommand{\rbdv}{r_{\bdv}}
\newcommand{\qadv}{{Q_\adv}}
\newcommand{\qbdv}{{Q_\bdv}}
\newcommand{\qro}{{Q_\ro}}
\newcommand{\qsim}{Q^{\simulator}}
\newcommand{\qroadv}[1]{Q^{\ro}_{#1}}
\renewcommand{\pf}{\p{f}}

\newcommand{\committer}{\pcalgostyle{C}}
\newcommand{\receiver}{\pcalgostyle{R}}

% Simulation extractability
\newcommand{\Qcom}{Q_\pcvarstyle{com}}
\newcommand{\Qev}{Q_\pcvarstyle{ev}}

%Plonk and Sonic
\newcommand{\ur}[1]{{#1\text{-}\mathsf{ur}}}

\newcommand{\plonk}{\ensuremath{\textnormal{\textsf{Plonk}}}}
\newcommand{\sonic}{{\ensuremath{\textnormal{\textsf{Sonic}}}}}
\newcommand{\groth}{\ensuremath{\textsc{Groth16}}}
\newcommand{\plonkmod}{\ensuremath{\plonk^\star}}
\newcommand{\plonkint}{\ensuremath{\plonk^\star}}
\newcommand{\polyprot}{\pcalgostyle{poly}}
\newcommand{\plonkintpoly}{\plonkint_\polyprot}
% \newcommand{\sonic}{\textsc{Sonic}}
\newcommand{\maxdegree}{\maxdeg}
\newcommand{\noofp}{\pcvarstyle{k}}
\newcommand{\noofev}{\pcvarstyle{k'}}
\newcommand{\encset}{E}
\newcommand{\PSc}{\PS_{com}}
\newcommand{\PHP}{\ensuremath{\pcschemestyle{\Gamma}}}
\newcommand{\indexer}{\pcalgostyle{I}}

\newcommand{\dlog}{\pcvarstyle{dlog}}
\newcommand{\ldlog}{\pcvarstyle{ldlog}}


%reductions
\newcommand{\extss}{\ext_\ss}
\newcommand{\extse}{\ext_\se}
\newcommand{\extt}{\ext_{\pcvarstyle{tree}}}
\newcommand{\compass}{\mathtt{C}}
\newcommand{\ks}{\pcvarstyle{ks}}
\renewcommand{\ss}{\pcvarstyle{ss}}
\newcommand{\rdvks}{{\rdv_\ks}}
\newcommand{\rdvs}{{\rdv_\pcvvarstyli{s}}}
\newcommand{\rdvdlog}{{\rdv_\dlog}}
\newcommand{\rdvldlog}{{\rdv_\ldlog}}
\newcommand{\rdvse}{{\rdv_\se}}
\newcommand{\rdvss}{{\rdv_\ss}}
\newcommand{\rdvur}{\rdv_\pcvarstyle{ur}}

\newcommand{\env}{\pcadvstyle{E}}
\newcommand{\zdv}{\pcadvstyle{Z}}

\newcommand{\advse}{\adv}
\newcommand{\advss}{{\adv_\ss}}

\newcommand{\epsss}{\eps_\ss}
\newcommand{\epsur}{\eps_\pcvarstyle{ur}}
\newcommand{\epsks}{\eps_{\pcvarstyle{ks}}}
\newcommand{\epsext}{\eps_{\pcvarstyle{ext}}}
\newcommand{\epsk}{\eps_{\pcvarstyle{k}}}
\newcommand{\epsbind}{\eps_\pcvarstyle{bind}}
\newcommand{\epsop}{\eps_\pcvarstyle{op}}
\newcommand{\epss}{\eps_\pcvarstyle{s}}
\newcommand{\epsbatch}{\eps_\pcvarstyle{btch}}
\newcommand{\epsdlog}{\eps_\dlog}
\newcommand{\epsldlog}{\eps_\ldlog}
\newcommand{\epsuber}{\eps_{\pcvarstyle{uber}}}
\newcommand{\epszk}{\eps_{\pcvarstyle{zk}}}
\newcommand{\epsev}{\eps_{\pcvarstyle{ev}}}
\newcommand{\epssr}{\eps_{\pcvarstyle{sr}}}
\newcommand{\plen}{\pcvarstyle{n}}

%errors
\newcommand{\err}{Err}
\newcommand{\errur}{\err_{ur}}
\newcommand{\errss}{\err_\ss}
\newcommand{\errfrk}{\err_\frkProb}


%forking
\newcommand{\forking}{\pcalgostyle{F}}
\newcommand{\genforking}{\pcalgostyle{GF}}

% markulf's
\newcommand{\Kzg}{\pcalgostyle{KZG}}

\def\cgen{{\pcalgostyle{ComGen}}}
\def\commit{{\pcalgostyle{Com}}}
\newcommand{\CM}{\commit}
\def\gk{{\pcvarstyle{gk}}}
\def\ck{{\pcvarstyle{ck}}}


\newcommand{\simgen}{\simulator\pcalgostyle{Gen}}
\newcommand{\simsample}{\simulator\pcalgostyle{Sample}}
\newcommand{\simopen}{\simulator\pcalgostyle{Open}}
\newcommand{\simprove}{\simulator\prover}

\newcommand{\challenger}{\pcalgostyle{Chal}}

% plonk constrains
\newcommand{\vql}{\vec{q_{L}}}
\newcommand{\vqr}{\vec{q_{R}}}
\newcommand{\vqm}{\vec{q_{M}}}
\newcommand{\vqo}{\vec{q_{O}}}
\newcommand{\vx}{\vec{x}}
\newcommand{\vqc}{\vec{q_{C}}}


%colors
\definecolor{darkmagenta}{rgb}{0.5,0,0.5}
\definecolor{lightmagenta}{rgb}{1,0.85,1}
\definecolor{lightmagenta}{rgb}{0.9,0.9,0.9}
\definecolor{darkred}{rgb}{0.7,0,0}
\definecolor{blueish}{rgb}{0.1,0.1,0.5}
\definecolor{greenish}{rgb}{0.1,0.4,0.1}
\definecolor{pinkish}{rgb}{0.9,0.8,0.8}
\definecolor{darkgreen}{rgb}{0,0.6,0}
\definecolor{lightgreen}{rgb}{0.85,1,0.85}
\definecolor{skyblue}{rgb}{0.3,0.9,0.99}


%comments
\DeclareRobustCommand{\markulf}[2]  {{\color{darkmagenta}\hl{\scriptsize\textsf{Markulf #1:} #2}}}
\DeclareRobustCommand{\michals}[2]  {{\color{blueish}\sethlcolor{pinkish}\hl{\scriptsize\textsf{Michal #1:} #2}}}
\DeclareRobustCommand{\antonio}[2]  {{\color{greenish}\sethlcolor{pinkish}\hl{\scriptsize\textsf{Antonio #1:} #2}}}
\newcommand{\task}[2]{\todo[author=\textbf{Task},inline]{({\textit{#1}}) #2}}
% \newcommand{\task}[2] {\xcommenti{Task}{#1}{#2}}
% \DeclareRobustCommand{\task}[2]  {{\color{black}\sethlcolor{yellow}\hl{\textsf{TASK #1:} #2}}}

%%% Local Variables:
%%% mode: latex
%%% TeX-master: "plonkext"
%%% End:


\title{On Simulation-Extractability of Polynomial Commitments}

\author{Michał Zając\inst{1}} 
%\iflncs{
\institute{Clearmatics, London, UK \\
\email{m.p.zajac@gmail.com}} 

\allowdisplaybreaks

\begin{document} \sloppy \maketitle

\begin{abstract} 
\end{abstract}

\section{Introduction}
\section{Preliminaries}
\subsection{Polynomial commitment scheme}
\subsection{Polynomial commitment.}
\label{sec:poly_com}
In the polynomial commitment scheme $\PCOM = (\kgen, \com, \open, \verify)$ the
committer $\committer$ can convince the receiver $\receiver$ that some
polynomial $\p{f}$ which $\committer$ committed to evaluates to $s$ at some
point $z$ chosen by $\receiver$.  The key generation algorithm $\kgen$ takes as
input a security parameter $\secparam$ and a parameter $\maxdeg$ which
determines the maximal degree of the committed polynomial. We assume that
$\maxdeg$ can be read from the output SRS.
%
% We require $\PCOM$ to have the following properties:
\begin{figure}[t!]
	\begin{pcvstack}[center,boxed]
		\begin{pchstack}
			\procedure{$\kgen(\secparam, \maxdeg)$}
			{
			\chi \sample \FF^2_p \\ [\myskip]
			\pcreturn \gone{1, \ldots, \chi^{\numberofconstrains + 2}}, \gtwo{\chi}\\ [\myskip]
				\hphantom{\hspace*{5.5cm}}	
        %\hphantom{\pcind \p{o}_i(X) \gets \sum_{j = 1}^{t_i} \gamma_i^{j - 1}
        %\frac{\p{f}_{i,j}(X) - \p{f}_{i, j}(z_i)}{X - z_i}}
      }
			
			\pchspace
			
			\procedure{$\com(\srs, \vec{\p{f}}(X))$}
			{ 
				\pcreturn \gone{\vec{c}} = \gone{\vec{\p{f}}(\chi)}\\ [\myskip]
				\hphantom{\pcind \pcif 
					\sum_{i = 1}^{\abs{\vec{z}}} r_i \cdot \gone{\sum_{j = 1}^{t_j}
					\gamma_i^{j - 1} c_{i, j} - \sum{j = 1}^{t_j} s_{i, j}} \bullet
				\gtwo{1} + }
			}
		\end{pchstack}
		% \pcvspace
    
		\begin{pchstack}
			\procedure{$\open(\srs, \vec{\gamma}, \vec{z}, \vec{s}, \vec{\p{f}}(X))$}
			{
			\pcfor i \in \range{1}{\abs{\vec{z}}} \pcdo\\ [\myskip]
      \pcind \p{o}_i(X) \gets \sum_{j = 1}^{t_i} \gamma_i^{j - 1}
      \frac{\p{f}_{i,j}(X) - \p{f}_{i, j}(z_i)}{X - z_i}\\ [\myskip] \pcreturn
      \vec{o} = \gone{\vec{\p{o}}(\chi)}\\ [\myskip]
				\hphantom{\hspace*{5.5cm}}	
			}
			
			\pchspace
			
			\procedure{$\verify(\srs, \gone{c}, \vec{z}, \vec{s}, \gone{\p{o}(\chi)})$}
			{
				\vec{r} \gets \FF_p^{\abs{\vec{z}}}\\ [\myskip]
				\pcfor i \in \range{1}{\abs{\vec{z}}} \pcdo \\ [\myskip]
				\pcind \pcif 
          \sum_{i = 1}^{\abs{\vec{z}}} r_i \cdot \gone{\sum_{j = 1}^{t_j}
          \gamma_i^{j - 1} c_{i, j} - \sum{j = 1}^{t_j} s_{i, j}} \bullet
          \gtwo{1} + \\ [\myskip] \pcind \sum_{i = 1}^{\abs{\vec{z}}} r_i z_i
          o_i
          \bullet \gtwo{1} \neq \gone{- \sum_{i = 1}^{\abs{\vec{z}}} r_i o_i }
          \bullet \gtwo{\chi} \pcthen  \\
					\pcind \pcreturn 0\\ [\myskip]
					\pcreturn 1.
			}
		\end{pchstack}
	\end{pcvstack}
	\caption{$\PCOMp$ polynomial commitment scheme.}
	\label{fig:pcomp}
  \end{figure}

\begin{figure}[t!]
	\begin{pcvstack}[center,boxed]
		\begin{pchstack}
			\procedure{$\kgen(\secparam, \maxdeg)$} {
				\alpha, \chi \sample \FF^2_p \\ [\myskip]
				\pcreturn \gone{\smallset{\chi^i}_{i = -\multconstr}^{\multconstr},
          \smallset{\alpha \chi^i}_{i = -\multconstr, i \neq
            0}^{\multconstr}},\\
        \pcind \gtwo{\smallset{\chi^i, \alpha \chi^i}_{i =
            -\multconstr}^{\multconstr}}, \gtar{\alpha}\\
				%\markulf{03.11.2020}{} \\
			%	\hphantom{\pcind \p{o}_i(X) \gets \sum_{j = 1}^{t_i} \gamma_i^{j - 1} \frac{\p{f}_{i,j}(X) - \p{f}_{i, j}(z_i)}{X - z_i}}
				\hphantom{\hspace*{5.5cm}}	
		}
			
			\pchspace
			
			\procedure{$\com(\srs, \maxconst, \p{f}(X))$} {
				\p{c}(X) \gets \alpha \cdot X^{\dconst - \maxconst} \p{f}(X) \\ [\myskip]
				\pcreturn \gone{c} = \gone{\p{c}(\chi)}\\ [\myskip]
				\hphantom{\pcind \pcif \sum_{i = 1}^{\abs{\vec{z}}} r_i \cdot
          \gone{\sum_{j = 1}^{t_j} \gamma_i^{j - 1} c_{i, j} - \sum{j = 1}^{t_j}
            s_{i, j}} \bullet \gtwo{1} + } }
		\end{pchstack}
		% \pcvspace
    
		\begin{pchstack}
			\procedure{$\open(\srs, z, s, f(X))$}
			{
				\p{o}(X) \gets \frac{\p{f}(X) - \p{f}(z)}{X - z}\\ [\myskip]
				\pcreturn \gone{\p{o}(\chi)}\\ [\myskip]
				\hphantom{\hspace*{5.5cm}}	
			}
			
			\pchspace
			
			\procedure{$\verify(\srs, \maxconst, \gone{c}, z, s, \gone{\p{o}(\chi)})$}
      {
        \pcif \gone{\p{o}(\chi)} \bullet \gtwo{\alpha \chi} + \gone{s - z
        \p{o}(\chi)} \bullet \gtwo{\alpha} = \\ [\myskip] \pcind \gone{c}
        \bullet \gtwo{\chi^{- \dconst + \maxconst}} \pcthen  \pcreturn 1\\
        [\myskip]
        \rlap{\pcelse \pcreturn 0.} \hphantom{\pcind \pcif \sum_{i =
            1}^{\abs{\vec{z}}} r_i \cdot \gone{\sum_{j = 1}^{t_j} \gamma_i^{j -
              1} c_{i, j} - \sum{j = 1}^{t_j} s_{i, j}} \bullet \gtwo{1} + } }
		\end{pchstack}
	\end{pcvstack}
	
	\caption{$\PCOMs$ polynomial commitment scheme.}
	\label{fig:pcoms}
\end{figure}
  
We emphasize the following properties of a secure polynomial commitment
$\PCOM$:
\begin{description}
\item[Evaluation binding:] A $\ppt$ adversary $\adv$ which outputs a commitment
  $\vec{c}$ and evaluation points $\vec{z}$ has at most negligible chances to open
  the commitment to two different evaluations $\vec{s}, \vec{s'}$. That is, let
  $k \in \NN$ be the number of committed polynomials, $l \in \NN$ number of
  evaluation points, $\vec{c} \in \GRP^k$ be the commitments, $\vec{z} \in
  \FF_p^l$ be the arguments the polynomials are evaluated at, $\vec{s},\vec{s}'
  \in \FF_p^k$ the evaluations, and $\vec{o},\vec{o}' \in \FF_p^l$ be the
  commitment openings. Then for every $\ppt$ adversary $\adv$
	\[
		\Pr
			\left[
			\begin{aligned}
				& \verify(\srs, \vec{c}, \vec{z}, \vec{s}, \vec{o}) = 1,  \\ 
				& \verify(\srs, \vec{c}, \vec{z}, \vec{s}', \vec{o}') = 1, \\
				& \vec{s} \neq \vec{s}'
			\end{aligned}
			\,\left|\,\vphantom{\begin{aligned}
                  & \\
                  & \\
                  &
                \end{aligned}}
			\begin{aligned}
				& \srs \gets \kgen(\secparam, \maxdeg),\\
				& (\vec{c}, \vec{z}, \vec{s}, \vec{s}', \vec{o}, \vec{o}') \gets \adv(\srs)
			\end{aligned}
			\right.\right] \leq \negl\,.
	\]

\end{description}
	
% We say that $\PCOM$ has the unique opening property if the following holds:
% \begin{description}
% \item[Opening uniqueness:] Let $k \in \NN$ be the number of committed
%   polynomials, $l \in \NN$ number of evaluation points, $\vec{c} \in \GRP^k$ be
%   the commitments, $\vec{z} \in \FF_p^l$ be the arguments the polynomials are
%   evaluated at, $\vec{s} \in \FF_p^k$ the evaluations, and $\vec{o} \in \FF_p^l$
%   be the commitment openings. Then for every $\ppt$ adversary $\adv$
% 	\[
% 		\Pr
% 			\left[
% 			\begin{aligned}
% 				& \verify(\srs, \vec{c}, \vec{z}, \vec{s}, \vec{o}) = 1,  \\ 
% 				& \verify(\srs, \vec{c}, \vec{z}, \vec{s}, \vec{o'}) = 1, \\
% 				& \vec{o} \neq \vec{o'}
% 			\end{aligned}
% 			\,\left|\, \vphantom{\begin{aligned}
%                   & \\
%                   & \\
%                   &
%                 \end{aligned}}
% 			\begin{aligned}
% 				& \srs \gets \kgen(\secparam, \maxdeg),\\
% 				& (\vec{c}, \vec{z}, \vec{s}, \vec{o}, \vec{o'}) \gets \adv(\srs)
% 			\end{aligned}
% 			\right.\right] \leq \negl\,.
% 	\]
% \end{description}
% Intuitively, opening uniqueness assures that there is only one valid opening
% for the committed polynomial and given evaluation point.

\begin{description}
\item[Commitment of knowledge]  For every $\ppt$ adversary $\adv$ who produces
  commitment $c$, evaluation point $z$, evaluation $s$ and opening $o$ there
  exists a $\ppt$ extractor $\ext$ such that
\[
  \Pr \left[
    \begin{aligned}
      & \p{f} = \ext_\adv(\srs, c),\\
      & c = \com(\srs, \p{f}),\\
      & \verify(\srs, c, z, s, o) = 1
    \end{aligned}
    \,\left|\,
      \vphantom{
        \begin{aligned}
          & \\
          & \\
          &
        \end{aligned}
        }
    \begin{aligned}
      & \srs \gets \kgen(\secparam, \maxdeg),\\
      & (c, z, s, o) \gets \adv(\srs)
    \end{aligned}
  \right.\right]
  \geq 1 - \epsk(\secpar).
\]
In that case we say that $\PCOM$ is $\epsk$-knowledge.
\end{description}
Intuitively when a commitment scheme is a commitment of knowledge then if an
adversary produces a (valid) commitment $c$, which it can open, then it also
knows the underlying polynomial $\p{f}$ which commits to that value.

\begin{definition}[Simulation extractable polynomial commitment]
  Let $\PCOM$ be a polynomial commitment scheme, let $\oracles$ be a simulator oracle
  which on input
  \begin{itemize}
  \item $(\msg{commit}, f)$ returns commitment $c = \com(f)$ and add $(f, c)$ to
    list $Q$;
  \item $(\msg{evaluate}, f, x)$ returns $f(x)$ if there is $c$ such that $(f,
    c) \in Q$; \michals{23.04}{This is not
      necessary}
  \item $(\msg{open}, c, x, y)$ returns an opening assuring that for a
    polynomial $f$, such that $c = \com(f)$, $f(x) = y$. If $c$ is not a
    commitment output by the oracle or $f(x) \neq y$, then ignore.
  \end{itemize}
  We say that $\PCOM$ is \emph{simulation extractable} if for any $\ppt$
  adversary $\adv$ with oracle access to $\oracles$ there exists extractor
  $\ext$ such that
  \[
    \Pr\left[
      \begin{aligned}
        & (\deg{f} > \maxdeg) \lor\\
        & (c \not\in \image(\com(f)))
        \end{aligned}
        \,\left|\,
          \begin{aligned}
          & \pp \sample \sgen(\secparam),
           \srs \sample \kgen(\pp, \maxdeg),\\
           & c \gets \adv^{\oracles}(\pp, \srs; r),
           (\cdot, c) \not\in Q, 
           f \gets \ext_{\adv}(\secparam, \pp, c, r)
        \end{aligned}
      \right.  \right] \leq \negl.
  \]
  \michals{23.04}{Can $\ext$ ask $\adv$ to give evaluations of the committed
    polynomial? That is how $\ext$ in a proof system works -- it evaluates
    polynomials submitted by the adversary on multiple challenges.}
  
\end{definition}

\subsection{Zero knowledge}
In a zero-knowledge proof system, a prover convinces the verifier of veracity of
a statement without leaking any other information. The zero-knowledge property
is proven by constructing a simulator that can simulate the view of a cheating
verifier without knowing the secret information---witness---of the prover. A
proof system has to be sound as well, i.e.~for a malicious prover it should be
infeasible to convince a verifier on a false statement. Here, we focus on proof
systems that guarantee soundness against $\ppt$ malicious provers.

More precisely, let $\RELGEN(\secparam) = \smallset{\REL}$ be a family of
$\npol$ relations. Denote by $\LANG_\REL$ the language determined by $\REL$. Let
$\prover$ and $\verifier$ be $\ppt$ algorithms, the former called \emph{prover}
and the latter \emph{verifier}. We allow our proof system to have a setup,
i.e.~there is a $\kgen$ algorithm that takes as input the relation description
$\REL$ and outputs a common reference string $\srs$. We denote by
$\ip{\prover(\REL, \srs, \inp, \wit)}{\verifier(\REL, \srs,\inp)}$ a
\emph{transcript} (also called \emph{proof}) $\zkproof$ of a conversation
between $\prover$ with input $(\REL, \srs, \inp, \wit)$ and $\verifier$ with
input $(\REL, \srs, \inp)$. We write
$\ip{\prover (\REL, \srs, \inp, \wit)}{\verifier(\REL, \srs, \inp)} = 1$ if in
the end of the transcript the verifier $\verifier$ returns $1$ and say that
$\verifier$ accepts it. We sometimes abuse notation and write
$\verifier(\REL, \srs, \inp, \zkproof) = 1$ to denote a fact that $\zkproof$ is
accepted by the verifier. (This is especially handy when the proof system is
non-interactive, i.e.~the whole conversation between the prover and verifier
consists of a single message $\zkproof$ sent by $\prover$).

A proof system $\proofsystem = (\kgen, \prover, \verifier, \simulator)$ for $\RELGEN$ is required to have three properties: completeness, soundness and zero knowledge, which are defined as follows:
\begin{description}
\item[Completeness.]
  An interactive proof system $\proofsystem$ is
  \emph{complete} if an honest prover always convinces an honest verifier, that
  is for all $\REL \in \RELGEN(\secparam)$ and $(\inp, \wit) \in \REL$
	\[
		\condprob{\ip{\prover (\REL, \srs, \inp, \wit)}{\verifier (\REL, \srs,
        \inp)} = 1}{\srs \gets \kgen(\REL)} = 1\,.
	\]
\item[Soundness.] We say that $\proofsystem$ for $\RELGEN$ is \emph{sound} if no
  $\ppt$ prover $\adv$ can convince an honest verifier $\verifier$ to accept a
  proof for a false statement $\inp \not\in\LANG$. More precisely, for all
  $\REL \in \RELGEN(\secparam)$
	\[
      \condprob{\ip{\adv(\REL, \srs, \inp)}{\verifier(\REL, \srs, \inp)} =
        1}{\srs \gets \kgen(\REL), \inp \gets \adv(\REL, \srs); \inp \not\in
        \LANG_\REL} \leq \negl\,;
	\]
\end{description}
Sometimes a stronger notion of soundness is required---except requiring that the
verifier rejects proofs of statements outside the language, we request from the
prover to know a witness corresponding to the proven statement. This property is
formalised by the so-called \emph{knowledge soundness}.
%\begin{description}
\begin{description}
\item[Knowledge soundness.]
We call an interactive proof system $\proofsystem$
\emph{knowledge-sound} if for any $\REL \in \RELGEN(\secparam)$ and a $\ppt$
adversary $\adv$
	% \begin{multline*}
	\[
	\Pr\left[
		\begin{aligned}
			& \verifier(\REL, \srs, \inp, \zkproof) = 1, \\
			& \REL(\inp, \wit) = 0
	 \end{aligned}
	  \,\left|\,
	 \begin{aligned}
		 & \srs \gets \kgen(\REL), \inp \gets \adv(\REL, \srs), \\
		 & (\wit, \zkproof) \gets \ext^{\ip{\adv(\REL, \srs, \inp)}{\verifier(\REL, \srs, \inp)}}(\REL, \inp)
	 \end{aligned}
	 \vphantom{\begin{aligned}
		 \adv (\zkproof) = 1, \\
		 \text{if $\zkproof{}$ is accepting} \\
		 \pcind \text{then $\REL(\inp, \wit)$}
	 \end{aligned}}\right.
	 \right] \leq \negl\,,
 % \end{multline*}
 \]
\end{description}

 % Usually the verifier verifies messages send by the prover by checking a number
 % of equations depend on the instance, SRS and the proof sent. These equations
 % are often called \emph{verification equations} and denoted $\vereq_i$, for $i$
 % being an index of the equation. It is usually required that an acceptable proof
 % yields $\vereq_i = 0$. In the proof systems we consider---$\plonk$ and
 % $\sonic$---verification equations can be seen as polynomials evaluated at the
 % trapdoor $\chi$. Thus, the verifier checks that $\vereq_i(\chi) =
 % 0$. Sometimes, we consider an \emph{idealised verifier},
 % cf.~\cite{EPRINT:GabWilCio19}, who instead of checking that polynomial
 % $\vereq_i(X)$ evaluates to $0$ at $\chi$ just checks that $\vereq_i(X)$ is a
 % zero polynomial.

 % \paragraph{Zero knowledge.}
\begin{description}
  \item[Zero knowledge.]
 We call a proof system $\proofsystem$ \emph{zero-knowledge} if for any
 $\REL \in \RELGEN(\secparam)$, $(\inp, \wit) \in \REL$, and adversary $\adv$
 there exists a $\ppt$ simulator $\simulator$ such that
	\begin{multline*}
	  \left\{\ip{\prover(\REL, \srs, \inp, \wit)}{\adv(\REL, \srs, \inp, \wit)}
      \,\left|\, \srs \gets \kgen(\REL)\vphantom{\simulator^\adv}\right.\right\} \approx_\secpar
		\left\{\simulator^{\adv}(\REL, \srs, \inp)\,\left|\, \srs \gets
        \kgen(\REL)\COMMENT{, (\inp, \wit) \gets \adv(\REL,
          \srs)}\vphantom{\simulator^\adv}\right.\right\}\,.
\end{multline*}
	%
We call zero knowledge \emph{perfect} if the distributions are equal and
\emph{computational} if they are indistinguishable for any $\nuppt$
distinguisher.

\end{description}
An SRS $\srs$ comes with a secret string called \emph{trapdoor} $\td$ that
allows the simulator to produce a simulated proof. In that case algorithm
$\kgen(\REL)$ outputs $(\srs, \td)$ and $\td$ is given to the simulator. In this
paper we distinguish simulators that requires a trapdoor to simulate and those
that do not. We call the former \emph{SRS-simulators} and denote them by
$\simulator_\td$.

% Occasionally, a weaker version of zero knowledge is sufficient. So called
% \emph{honest verifier zero knowledge} (HVZK) assumes that the verifier's
% challenges are picked at random from some predefined set. Although weaker, this
% definition suffices in many applications. Especially, an interactive
% zero-knowledge proof that is HVZK and \emph{public-coin} (i.e.~the verifier
% outputs as challenges its random coins) can be made non-interactive and
% zero-knowledge in the random oracle model by using the Fiat--Shamir
% transformation.

% \paragraph{Idealised verifier and verification equations}
% Usually the verifier $\verifier$ verifies messages sent by the prover by
% checking a number of equations that depend on the instance, SRS and the proof. We
% call these equations \emph{verification equations} and denote by $\vereq_i$, for
% $i$ being an index of the equation. We require that an acceptable
% proof yields $\vereq_i = 0$.

% In the proof systems we consider---$\plonk$ and $\sonic$---verification
% equations can be interpreted as group representations of polynomial evaluations
% at the trapdoor $\chi$. In other words, the verifier checks that
% $\vereq_i(\chi) = 0$. Gabizon et al.~\cite{EPRINT:GabWilCio19} formalized a
% notion of an \emph{idealised verifier} who, instead of checking that polynomial
% $\vereq_i(X)$ evaluates to $0$ at $\chi$, just checks that $\vereq_i(X)$ is a
% zero polynomial.

\begin{definition}[Polynomial IOP,~\cite{EPRINT:Szepieniec20}]
  Let $\REL$ be an indexed relation with a corresponding language $\LANG$, $\FF$
  some finite field, and $\maxdeg$ a degree bound and $\noofp$ a parameter. A
  \emph{polynomial IOP for $\REL$ with degree bound $\maxdeg$} is a pair of
  interactive machines $\prover, \verifier$ such that
\begin{itemize}
\item $(\prover, \verifier)$ is an interactive proof for $\LANG$ with $r$ rounds
  and soundness error $\epss$;
\item $\prover$ sends to $\verifier$ polynomials $f_i \in \FF[X]$,
  $i \in \range{1}{\noofp}$, of degree at most $\maxdeg$;
\item $\verifier$ is an oracle machine with access to a list of oracles, which
  contains one oracle for each polynomial it has received from the prover;
\item When an oracle associated with a polynomial $f_i(X)$ is queried on a point
  $z_j \in \FF$, the oracle responds with the value $f_i(z_j)$; 
\item $\verifier$ sends challenges $\alpha_k \in \FF$ to $\prover$;
\item $\verifier$ is public coin.
\end{itemize}
\end{definition}

\michals{28.04}{Add preprocessing and zero knowledge}

\begin{definition}[Witness encoding PIOP (WEPIOP)]
  We say that a PIOP $\PS$ is \emph{witness encoding} if there is a pair of
  function $\decode$ such that for any $(\inp, \wit) \in \REL$ and polynomials
  $f_1, \ldots, f_\noofp$ sent by an honest prover,
  $\decode(f_1, \ldots, f_\noofp) = \wit$.
\end{definition}
In other words, PIOP is witness encoding if for any valid proof for a statement
$\inp$ in the language, the corresponding witness can be read from the
polynomial coefficients. We note that this is the case for virtually all
PIOPs. \michals{28.04}{Check!}]

\begin{definition}[Ordered WEPIOP]
  We say that a WEPIOP $\PS$ is \emph{ordered} if a proof for a statement
  $\inp$, and witness $\wit$ can be divided into the following stages
  \begin{description}
  \item[Witness encoding phase] where the prover sends polynomials $f_1, \ldots,
    f_{\noofp'}$ which encode the witness. 
  \item[Challenge phase]  where the verifier either
    \begin{itemize}
    \item sends its challenges which are answered by the prover by uploading
      polynomials $f_{\noofp'}, \ldots, f_{\noofp}$ such that $f_1, \ldots
      f_{\noofp}$ are determined using previously uploaded polynomials and
      verifier's challenges and queries; and / or
    \item the verifier submits its queries to the polynomial oracles
      $\oracleo(f_1), \ldots, \oracleo(f_\noofp)$.
    \end{itemize}
  \end{description}
  Furthermore, the verifier queries all witness-encoding polynomials.
\end{definition}

\begin{lemma}
  Ordered WEPIOP has unique-response property after witness-encoding phase.
\end{lemma}
\begin{proof}
  
\end{proof}

\section{SE PoK from SE PC}
\subsection{Result for polynomial IOPs}
\michals{28.04}{One thing is to show that using FS gives simulation
  extractability of a polynomial IOP, another is to show that a compiler (using
  commitments instead of polynomial oracles) preserves this property.}
\begin{lemma}
  Let $\PS$ be a zero-knowledge WEPIOP proof system and $\PS_\fs$ be the same system after
  applying Fiat--Shamir transform. Then $\PS_{\fs}$ is simulation-extractable.
\end{lemma}
\begin{proof}
  
\end{proof}

\subsection{Result for polynomial protocols}
\begin{lemma}
  Let $\PCOM$ be a simulation extractable polynomial commitment scheme. Let
  $\PS$ be a zero-knowledge polynomial proof system such that no adversary (even
  unbounded) can tell a real proof from a simulated one with advantage greater
  than $\epszk$. Then for every $\ppt$ adversary $\adv$ there exists extractor
  $\ext$ which extracts every polynomial sent in the proof. More precisely,
  probability
  \[
    \Pr\left[
      \begin{aligned}
        & \vec{f} \gets \ext_\adv(\REL, \srs, \vec{c}, \zkproof) \land \\
        & (\vec{c} \not\in \image(\vec{f}) \lor \exists i: \deg(f_i) > \maxdeg)
        \end{aligned}
      \, \left| \,
        \begin{aligned}
          & \srs \sample \kgen(\REL), \\
          & \zkproof = ((c_1, \vec{x_1}, \vec{y_{1}}), \ldots, (c_\plen,
          \vec{x}_\plen, \vec{y}_{\plen})) \gets \adv^{\simulator}(\REL, \srs)
        \end{aligned}
        \right.
    \right] \leq \negl
  \]
\end{lemma}
\begin{proof}
  Assume the contrary. That is, assume that there is a $\ppt$ adversary $\adv$
  producing the proof
  $\zkproof = (c_1, \vec{x_1}, \vec{y_1}, \ldots, c_n, \vec{x_\plen}, \vec{y_\plen})$
  \michals{23.04}{Note that $x$ and $y$ are vectors as in the proof polynomials
    hidden in $c_i$-s may be evaluated at multiple points.} such that any
  extractor $\ext$ probability that $\ext$ fails to return polynomials
  $f_1, \ldots f_\plen$, which commitments are $c_1, \ldots c_\plen$, is at least $\nu$,
  for some non-negligible $\nu$. We use $\adv$ to build an adversary $\bdv$ that
  breaks simulation-extractability of the polynomial commitment scheme.

  Let $\cdv$ be an (inefficient) challenger for $\bdv$ that prepares an instance
  of a problem, i.e.~it picks public parameters $\pp$, generates the polynomial
  commitment scheme SRS, provides adversary $\bdv$ with it and answers
  adversary's queries to the simulator query $\oracles$. $\cdv$ also picks an
  extractor $\ext_\bdv$ for $\bdv$. Eventually, $\bdv$ outputs a commitment $c$
  to some polynomial and $\cdv$ checks whether $\ext_\bdv$ is able to find a
  polynomial $f$ of degree less than $\maxdeg$ in the image of $\com$. If
  $\ext_\bdv$ succeeds then $\cdv$ outputs $0$ ($\bdv$ looses), otherwise it
  outputs $1$ ($\bdv$ wins). We show a strategy for $\bdv$ which guarantees
  probability of winning at least $(1 - \epszk) \cdot \infrac{\nu}{n}$.

  Adversary $\bdv$ begins with generating an SRS $\srs_{\PS}$ of $\PS$ using
  $\srs_{\PCOM}$ \michals{26.04}{describe how both SRS-s are related. That should
    be a property of the proof system.} and providing $\adv$ with it. Then it
  picks at random index $i$, which is $\bdv$'s guess for which of the
  polynomials is not going be extracted by the extractor $\ext$. Note that
  probability that $\bdv$ picks the right $i$ is at least $\infrac{1}{n}$.  Then
  $\bdv$ starts $\adv(\REL, \srs_{\PS})$ and answers $\adv$'s queries to the
  simulation oracle $\oracles$. Since $\bdv$ may not know the SRS's trapdoor, it
  provides $\adv$ with \emph{real} proofs, not simulated ones. However, this
  does not change the probability of $\adv$ returning a
  simulation-extractability breaking proof more than $\epszk$ as otherwise such
  $\adv$ could be used to break the zero-knowledge property.

  Eventually $\adv$ returns a proof $\zkproof = ((c_1, \vec{x_1}, \vec{y_1}),
  \ldots, (c_\plen, \vec{x_\plen}, \vec{y_\plen}))$, $\bdv$ gets commitment
  $c_i$ and outputs it. \michals{26.04}{Maybe the property of extractability
    could be a bit weaker --- say the adversary cannot output a valid commitment
  which he can open (at which point?) without knowing the underlying
  polynomial. We should also consider this case. The stronger extractability
  property may be difficult to meet.} Note that the extractor $\ext_\bdv$ is not able to
reveal the polynomial behind $c_i$ with an overwhelming probability. Assume
otherwise, then $\bdv$ could be used to reduce the probability of $\adv$'s
success. More precisely, \hl{continue}

\end{proof}

\begin{lemma}
  \hl{If we can extract all polynomials committed in the proof, then we can
    extract the witness}
\end{lemma}
\begin{proof}
  
\end{proof}

\begin{corollary}
  Let $\PCOM$ be a simulation extractable polynomial commitment scheme. Let
  $\PS$ be a zero-knowledge polynomial proof system, then $\PS$ is simulation extractable.
\end{corollary}
\begin{proof}
  Let $\adv$ be a simulation-extractability adversary. Let $\extpcom_\adv$ be a
  polynomial commitment scheme extractor for $\adv$. We build a $\PS$ extractor
  $\extps_\adv$ using $\extpcom_\adv$ as a building block.
\end{proof}

\bibliographystyle{alpha}
\bibliography{cryptobib/abbrev1,cryptobib/crypto}

\end{document}